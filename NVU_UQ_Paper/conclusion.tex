\section{Conclusion}
 A three stage methodology is presented for global sensitivity analysis of numerical cell models with a large number of parameters. To the authors knowledge, this is the first type of analysis which investigates neuroscience models of such size. The analysis investigated three quantities of interest pertaining to a numerical model of neurovascular coupling. The results indicated several prevalent features of the model. A significant influence of the persistent $Na^+$ channel activation variable on the average extracellular space $K^+$, a two parameter set (the inwardly rectifying $K^+$ channel shift parameter and the index for the cytosolic $Ca^{2+}$) which characterizes most of the variability in the average volumetric flow rate, and strong similarities between the most influential parameters for the average volumetric flow rate and minimum value of the combined actin/myocin complex.
 
In addition to the results reported in this article, four other QoI's were considered. Two of them, the maximum and average potassium concentration in the Astrocyte, were omitted because the parameter to QoI mapping is nearly constant and hence global sensitivity analysis is not necessary. Specifically, the mean of the QoI is approximately 37 times larger than its standard deviation for both cases. The other two unreported QoIs correspond to lag times. The first being the duration of time between the application of the stimulus and the minimum of the phosphorylated actin myosin complex, and the second being the duration of time between the application of the stimulus and the maximum of the radius. In both cases, the QoI exhibited highly nonlinear behavior which we were unable to approximate with linear or PC surrogates trained on the existing data. In fact, fitting such nonlinearities would likely require more samples than is computationally feasible for this model. The linear surrogate had 59\% and 47\% relative $L^2$ errors for these two QoIs, respectively. Our global sensitivity analysis methodology was unsuccessful because the linear surrogate was an unreliable tool for screening. Defining the QoI as the maximum/minimum value instead of the time lag makes the analysis more tractable. These maximum/minimum value QoIs were also considered and yielded similar results to the QoIs reported in the article.

For a given model and collection of samples, the methodology presented in this article may be applicable for some QoIs and intractable for others. The success of our method depends upon the surrogate models being sufficiently accurate. A general principle is that QoIs defined as averages will be more amenable for analysis than, for instance, minimum values or lag times. A practical benefit of our method is that any QoI may be considered without requiring additional model evaluations. The sampling and ODE solves are executed once, followed by computing the QoIs and performing global sensitivity analysis, which may be easily repeated for many different QoIs.
 
 
% Results showed that for the QoI defined by the average value of extracellular space $K^+$, the persistent $Na^+$  channel activation variable provided an approximately combined  62 \% of the variation in $K^+$ whilst the $K^+$ leak channel provided 21 \% and the KDR channel 11 \%. The dendrite length although in the first 5 ranked parameters provided only 3 \% variation. All other parameters were considered unimportant. \\
% For the second QoI (average value of the volumetric flow rate), the main variability ( 46 \%) was determined by the inwardly rectifying $K^+$ channel parameter which shifs the channel conductance to the right, whilst the index for the cytosolic $Ca^{2+}$ provided an approximately 33 \% influence. Three other parameters provided only a small influence but were associated with either the $K_{IR}$  or the BK $K^+$ channel conductances. 
% Finally the third QoI concerned the minimum value of the combined actin/myocin complex. In a related fashion to the volumetric flow rate, similar influential parameters appeared in the ranked list, the only two non-repeating parameters between the flow rate QoI and the actin/myson complex was the leak conductance GKi and $z_3$, which, as noted above, have very little effect. \\
% We should point out some limitations of the methodology in that other QoIs proved difficult if not impossible to analyse due to the resulting skewed distribution functions of the QoIs. Future work will investigate other methods which alleviate this difficulty. 
% It is important to note that the analysis itself is not one from which a more simple model of neurovascular coupling should be derived. In fact because different functions of the system are determined by a number of pathways (some of them independent) the numerical model in and of itself cannot be reduced to a mathematically analysable form. Indeed to investigate neurophysiological problems both from pathological and "normal" states models will very probably need to gain in complexity. The sensitivity analysis presented here however provides a searchlight on important and influential aspects of the model and hence the physiology. \\
% Attention should now be drawn towards how one can characterise which QoIs can in fact be analysed. 
%victory dance because we are the first people to do this type of problem.
%
% Forward looking paragraph, maybe multiscale modelling. 
%
%
%Look at other QoIs mention the fact that things don't work for every QoI future work how far can we push. How to characterise which QoIs we can deal with. 