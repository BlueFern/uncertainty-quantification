\section{Conclusion}
 A three stage global analysis methodology is presented for numerical cell models with a large number of parameters. It is believed that this is the first type of analysis which investigates models of such size. The analysis investigated three quantities of interest pertaining to a numerical model of neurovascular coupling. Results showed that for the QoI of the average value of extracellular space $K^+$, the persistent $Na^+$  channel activation variable provided a combined approximately 62 \% of the variation in $K^+$ whilst the $K^+$ leak channel provided 21 \% and the KDR channel 11 \%. The dendrite length although in the first 5 ranked parameters provided only 3 \% variation. All other parameters were considered unimportant. \\
 For the second QoI (average value of the volumetric flow rate), the main variability ( 46 \%) was determined by the inwardly rectifying $K^+$ channel parameter which shifs the channel conductance to the right, whilst the index for the cytosolic $Ca^{2+}$ provided approximately 33 \% influence. Three other parameters provided only a small influence but were associated with either the $K_{IR}$  or the BK $K^+$ channel conductances. 
 Finally the third QoI concerned the minimum value of the combined actin/myocin complex. In a related fashion to the volumetric flow rate, similar influential parameters appeared in the ranked list, the only two non-repeating parameters between the flow rate QoI and the actin/myson complex was the leak conductance GKi and $z_3$, which, as noted above, have very little effect. \\
 We should point out some limitations of the methodology in that other QoIs proved difficult if not impossible to analyse due to the resulting skewed distribution functions of the QoIs. Future work will investigate other methods which alleviate this difficulty. 
 The analysis itself is not one from which a more simple model of neurovascular coupling should be derived. In fact because different functions of the system are determined by a number of pathways (some of them independent) the numerical model in and of itself cannot be reduced to a mathematically analysable form. Indeed to investigate neurophysiological problems both from a pathological and "normal" state models will very probably need to gain in complexity. The sensitivity analysis presented here however provides a searchlight on important and influential aspects of the model and hence the physiology. \\
 Attention should now be drawn towards how one can characterise which QoIs can in fact be analysed. 
%victory dance because we are the first people to do this type of problem.
%
% Forward looking paragraph, maybe multiscale modelling. 
%
%
%Look at other QoIs mention the fact that things don't work for every QoI future work how far can we push. How to characterise which QoIs we can deal with. 