\section{Methodology}\label{sec:meth}

Our analysis may be described in a four step process: (i) define a probability distribution for $\theta$ which models its uncertainty, (ii) draw samples from this distribution, (iii) evaluate the QoI for each of these parameter samples, (iv) use these QoI evaluations to infer its sensitivity to each parameter. Subsection~\ref{sec:param_dist_fit} describes our approach for step (i). Step (ii) is easily executed using standard methods. The computational bottleneck for our analysis is the QoI evaluations (which require ODE solves) in step (iii). While relatively large and stiff, the ODE system (\ref{caboodle}) can be  solved with standard tools and methods, here through the MATLAB routine ode15s with relative and absolute tolerances of $10^{-4}$. The evaluation of the above QoIs themselves from the ODE solutions is straightforward and can be done at low cost. Inferring the sensitivities may be done in a plurality of ways. Because of the parameter dimension and computational cost of model evaluations, we adopt a multi-phased procedure of screening, surrogate modeling, and computing Sobol' indices, described in Subsections~\ref{sec:screen}, ~\ref{sec:surrogate}, and ~\ref{sec:gsa} respectively. We summarize our method in Subsection~\ref{sec:summary}.

\subsection{Parameter Distribution Fitting}
\label{sec:param_dist_fit}

Describing the uncertainty attached to the parameter vector $\boldsymbol{\theta}$ is an important and delicate {\sl modeling assumption}. Here, we give each $\theta_i$, $i=1,\dots, 160$, a nominal value $\bar \theta_i$ and assume each parameter to be independent and uniformly distributed over the interval $[0.9\, \bar\theta_i, 1.1 \,\bar\theta_i]$, i.e., within $\pm 10\%$ of the nominal value. Larger intervals where considered for the parameters. For the NVU model under consideration, increasing the width of the interval resulted in many samples for which the solution exhibited atypical or non-physical behavior, or in some cases ode15s was unable to solve the system. Our choice of $\pm 10\%$ uncertainty is considered a reasonable compromise between accounting for uncertainty and ensuring computational feasibility. 
	
Our assumption of parameter independence is incorrect. In fact, ode15s is unable to solve the system for most samples drawn under this assumption. The parameter dependencies, which are unknown a priori, are discovered through a computational procedure akin to Approximate Bayesian Computation \cite{abc}. Our approach, described below, is computationally intractable when applied to the NVU model under consideration. We simplify the model by removing the stimulus, i.e. modeling the steady state behavior of the system. A collection of $S$ samples, $\boldsymbol{\theta}^k$, $k=1,2,\dots,S$, are drawn from our initial distribution (assuming independence) and the ODE system is solved for each parameter sample. The collection of solutions are post-processed to partition the samples $\{\boldsymbol{\theta}^k\}_{k=1}^S$ into two subsets $\{\boldsymbol{\theta}^{a_k}\}_{k=1}^{S_a}$ and $\{\boldsymbol{\theta}^{r_k}\}_{k=1}^{S_r}$, where $a_k$, $k=1,2,\dots,S_a$ denotes the samples where the steady state solution exhibited normal behavior and $r_k$, $k=1,2,\dots,S_r$ denotes the samples where it does not. We reject the samples $\{\boldsymbol{\theta}^{r_k}\}_{k=1}^{S_r}$ and fit a distribution to the accepted samples $\{\boldsymbol{\theta}^{a_k}\}_{k=1}^{S_a}$ using standard statistical methods. This new distribution is sampled $S$ times,  the ODE system is solved for each sample, and the results are post-processed into accepted and rejected samples again. We continue this process iteratively until satisfactory convergence of the fitted distribution.

\subsection{Screening}
\label{sec:screen}
Having determined a parameter distribution, we evaluate the QoI $q=g(\boldsymbol{\theta})$ at $M$ different samples, denote them $\boldsymbol{\theta}^k$, $k=1,2,\dots,M$.
We fit a {\sl linear model} to the QoI under study
\begin{eqnarray}
g(\boldsymbol\theta^k) = g(\theta_1^k, \dots, \theta_P^k) \approx \beta_0 + \sum\limits_{j=1}^{160} \beta_j \theta_j^k, \quad k=1, \dots, M. \label{lr}
\end{eqnarray}
This approach yields a crude (but highly efficient here) sensitivity analysis of the model with respect to the $\theta_j$'s, $j=1,\dots, 160$. We assign a preliminary importance measure to each $\theta_j$ by computing for each of them the relative size of their coefficient in the above linear approximation, i.e.
\begin{eqnarray*}
L_j = \frac{\vert \beta_j \vert}{\sum\limits_{\ell=1}^{160} \vert \beta_\ell \vert}, \qquad j=1,\dots,160.
\end{eqnarray*}
To obtain a model with a more manageable size, we reduce the parameter space to only the $\theta_j$'s for which $L_j>0.01$. We denote these $r$ parameters $\{ \theta_{j_i}\}_{i=1}^r$. The rest of the parameters are regarded as non-influential and treated as latent variables, even though they are uncertain, their specific values (within the given range) have little bearing of the considered QoI. In other words, we consider the approximation 
\begin{eqnarray}
g(\theta_1, \dots, \theta_{160}) \approx h(\theta_{j_1}, \dots, \theta_{j_r}), \label{reddim}
\end{eqnarray}
where $h$ is obtained from $g$ by treating the non-influential parameters as latent. In Section~\ref{sec:results}, this reduction yields around 15-20 parameters instead of the original 160. We use $\hat{\boldsymbol{\theta}}$ to denote the reduced parameter vector.

\subsection{Surrogate model}
\label{sec:surrogate}
For any of the three considered QoIs, our information on the function $h$ defined (\ref{reddim}) consists of the set of sampled values $\{ h(\theta_{j_1}^k, \dots, \theta_{j_r}^k)\}$, $k=1, \dots, M$.  To facilitate the use of standard GSA tools, which may require derivatives or variance estimations, it is both convenient and computationally advantageous to construct an approximating function, i.e., a surrogate model. 
We use a sparse Polynomial Chaos (PC) surrogate. This amounts to introducing a polynomial approximation of $h$ of the type
\begin{eqnarray}
h(\hat{\boldsymbol{\theta}}) \approx H(\hat{\boldsymbol{\theta}}) \equiv \sum_{\boldsymbol{\alpha}} c_{\boldsymbol{\alpha}} \psi_{\boldsymbol{\alpha}}(\hat{\boldsymbol{\theta}}) \label{pce}
\end{eqnarray}
where the $\psi_{\boldsymbol{\alpha}}$'s are multivariate polynomials which are orthogonal with respect to the probability distribution function (PDF) $p_{\hat{\boldsymbol{\theta}}}$ of $\hat{\boldsymbol{\theta}}$, i.e.
\begin{eqnarray}
\int \psi_{\boldsymbol{\alpha}}(\mathbf x) \psi_{\boldsymbol{\beta}}(\mathbf x)\, p_{\hat{\boldsymbol{\theta}}}(\mathbf x) \, d\mathbf{x} = \delta _{\boldsymbol{\alpha},\boldsymbol{\beta}} \label{ortho}
\end{eqnarray}
where $\boldsymbol{\alpha}$ and $\boldsymbol{\beta}$ are multi-indices and $\delta _{\boldsymbol{\alpha},\boldsymbol{\beta}}$ is the generalized Kronecker symbol. The coefficients are computed through least-squares minimization, see the Appendix for additional discussion. All surrogate models are validated using 10-fold cross validation.

Polynomial Chaos is by now a well documented method. We use the \textit{UQLab} implementation for the results below and refer the reader to its manual \cite{uqlab} for more details.


\subsection{Sobol' indices} 
\label{sec:gsa}
We use variance based GSA to assess the relative importance of the input parameters of $H$ in (\ref{pce}). In their simplest form, the total Sobol' indices \cite{saltellitotalindex} apportion to uncertain parameters, or sets thereof, their relative contribution to the variance of the output. Indeed, thanks to the law of total variance, we can decompose the variance of $q = H(\hat{\boldsymbol{\theta}})$ as follows
\begin{eqnarray}
\operatorname{var}(q) = \operatorname{var}(\mathbb E[q|\hat{\boldsymbol{\theta}}_{\sim i}]) + \mathbb E[\operatorname{var}(q|\hat{\boldsymbol{\theta}}_{\sim i})], \label{ltv}
\end{eqnarray}
where $\hat{\boldsymbol{\theta}}_{\sim i}$ denotes all the parameters in $\hat{\boldsymbol{\theta}}$ except $\hat{\boldsymbol{\theta}}_i$. If we assume now that all the input parameters of $H$ are known with certainty, i.e,, if $\hat{\boldsymbol{\theta}}_{\sim i}$ is known, then the remaining variance of $q$ is simply given by 
\begin{eqnarray*}
\operatorname{var}(q) - \operatorname{var}(\mathbb E[q|\hat{\boldsymbol{\theta}}_{\sim i}]) = \mathbb E[\operatorname{var}(q|\hat{\boldsymbol{\theta}}_{\sim i})]. 
\end{eqnarray*}
This latter expression is thus a natural way of measuring how influential $\hat{\boldsymbol{\theta}}_i$ is; the corresponding total Sobol' index $S_{T_i}$ is but a scaled version of this
\begin{eqnarray}
S_{T_i} = \frac{\mathbb E[\operatorname{var}(q|\hat{\boldsymbol{\theta}}_{\sim i})]}{\operatorname{var}(q) }. \label{sobol}
\end{eqnarray}
From this definition, one easily observes that $S_{T_i} \in [0,1]$; large values indicate that $\hat{\boldsymbol{\theta}}_i$ is important, $S_{T_i} \approx 0$ implies that $\hat{\boldsymbol{\theta}}_i$ is not important.
The relevance of this basic definition can be extended to time dependent QoIs \cite{timegsa}, stochastic QoIs \cite{stogsa} or correlated parameters \cite{corvar}.


\subsection{Summary of the Method}
\label{sec:summary}
Algorithm~\ref{algo} provides a summary of the method. 

\begin{algorithm}
\caption{overall numerical approach}\label{algo}
\begin{algorithmic}[1]
\While{parameter distribution has not converged}
\For{$k=1:S$} \Comment{sampling}
\State sample parameter distribution  $\longrightarrow \boldsymbol{\theta}^k$; solve (\ref{caboodle}) (without stimulus) $\longrightarrow \mathbf y^k$
\EndFor
\State partition the parameter samples into accepted and rejected samples
\State fit a new distribution to the accepted samples \Comment{see \S~\ref{sec:param_dist_fit}}
\EndWhile
\For{$k=1:M$} \Comment{sampling (final distribution)}
\State sample parameter distribution  $\longrightarrow \boldsymbol{\theta}^k$; solve (\ref{caboodle}) (with stimulus) $\longrightarrow \mathbf y^k$
\EndFor
\For{each QoI $q$}
\State solve the least-squares problem (\ref{lr}) \Comment{linear model}
\State identify the influential parameter vector $\hat{\boldsymbol\theta} =   (\theta_{j_1}, \dots, \theta_{j_r})$  \Comment{screening}
\State fit the polynomial chaos surrogate $H$ \eqref{pce} \Comment{surrogate model}
\State compute total Sobol' indices (\ref{sobol}) of the surrogate model $H$ \Comment{Sobol' indices}
\EndFor
\end{algorithmic}
\end{algorithm}




