\section{Methodology}\label{sec:meth}

How to describe the uncertainty attached to the parameter vector $\boldsymbol{\theta}$ is an important and delicate {\sl modeling assumption}. Here, we give each $\theta_i$, $i=1,\dots, 160$, a nominal value $\bar \theta_i$ and assume each parameter to be uniformly distributed over the range of value $[0.9\, \bar\theta_i, 1.1 \,\bar\theta_i]$, i.e., within $\pm 10\%$ of the nominal value. All the parameters  are assumed to be independent, an assumption we revisit and improve upon below. 

A key  step in analyzing the dependency of a QoI $q = g(\boldsymbol{\theta})$ on its inputs is the evaluation of $g$ at $M$ samples $\boldsymbol\theta^k$, $k=1,\dots, M$, from the parameter distribution (where superscripts are used to refer to specific samples). 
This is also the  main  computational bottleneck as it requires the numerical resolution of  the ODE system (\ref{caboodle})  by computing the $M$ solutions $\mathbf y^k$, $k=1,\dots , M$, corresponding to  the parameter samples. 
While relatively large and stiff, the ODE system (\ref{caboodle}) can be  solved with standard tools and methods, here through the MATLAB routine ode15s  with relative and absolute tolerances of $10^{-4}$. The evaluation of the above QoIs themselves from the ODE solutions is straightforward and can be done at low cost.



For the present application, the large parameter dimension and computational cost of the above model evaluations prohibit the direct use of most global sensitivity analysis methods. In other words, the value of $M$ (number of samples)  we can ``afford" is too small for these methods to be applicable. 
For each QoI, the overall procedure is thus as follows (i) dimension reduction, (ii) surrogate modeling, (iii) global sensitivity analysis.
These steps are individually considered in the next three subsections. 




\subsection{Dimension reduction}
Using the previously generated model evaluations, we fit a {\sl linear model} to a QoI under study
\begin{eqnarray}
g(\boldsymbol\theta^k) = g(\theta_1^k, \dots, \theta_P^k) \approx \beta_0 + \sum\limits_{j=1}^{160} \beta_j \theta_j^k, \quad k=1, \dots, M. \label{lr}
\end{eqnarray}
This approach yields a crude (but highly efficient here) sensitivity analysis of the model with respect to the $\theta_j$'s, $j=1,\dots, 160$. We assign a preliminary importance measure to each $\theta_j$ by computing for each of them the relative size of their coefficient in the above linear approximation, i.e.
\begin{eqnarray*}
L_j = \frac{\vert \beta_j \vert}{\sum\limits_{\ell=1}^{160} \vert \beta_\ell \vert}, \qquad j=1,\dots,160.
\end{eqnarray*}
To obtain a model with a more manageable size, we reduce the parameter space to only the $\theta_j$'s for which $L_j>0.01$. We denote these $r$ parameters $\{ \theta_{j_i}\}_{i=1}^r$. The rest of the parameters are regarded as non-influential and set to their nominal values since, even though they are uncertain, their specific values (within the given range) have little bearing of the considered QoI. In other words, we consider the approximation 
\begin{eqnarray}
g(\theta_1, \dots, \theta_{160}) \approx h(\theta_{j_1}, \dots, \theta_{j_r}), \label{reddim}
\end{eqnarray}
where $h$ is obtained from $g$ by fixing any non-influential parameter $\theta_j$ in $g$  to its nominal value $\bar\theta_j$. In Section~\ref{sec:results}, this reduction yields around 15-20 parameters instead of the original 160. We denote by $\hat{\boldsymbol{\theta}}$ the reduced parameter vector. 

\subsection{Surrogate model}
For any of the three considered QoIs, our information on the function $h$ defined (\ref{reddim}) consists of the set of sampled values $\{ h(\theta_{j_1}^k, \dots, \theta_{j_r}^k)\}$, $k=1, \dots, M$.  To facilitate the use of standard GSA tools, which may require derivatives or variance estimations, it is both convenient and computationally advantageous to construct an approximating function, i.e., a surrogate model. 
We use a sparse Polynomial Chaos (PC) surrogate. This amounts to introducing a polynomial approximation of $h$ of the type
\begin{eqnarray}
h(\hat{\boldsymbol{\theta}}) \approx H(\hat{\boldsymbol{\theta}}) \equiv \sum_{\boldsymbol{\alpha}} c_{\boldsymbol{\alpha}} \psi_{\boldsymbol{\alpha}}(\hat{\boldsymbol{\theta}}) \label{pce}
\end{eqnarray}
where the $\psi_{\boldsymbol{\alpha}}$'s are multivariate polynomials which are orthogonal with respect to the probability distribution function (PDF) $p_{\boldsymbol{\theta}}$ of $\boldsymbol{\theta}$, i.e.
\begin{eqnarray}
\int \psi_{\boldsymbol{\alpha}}(\mathbf x) \psi_{\boldsymbol{\beta}}(\mathbf x)\, p_{\boldsymbol{\theta}}(\mathbf x) \, d\mathbf{x} = \delta _{\boldsymbol{\alpha},\boldsymbol{\beta}} \label{ortho}
\end{eqnarray}
where $\boldsymbol{\alpha}$ and $\boldsymbol{\beta}$ are multi-indices and $\delta _{\boldsymbol{\alpha},\boldsymbol{\beta}}$ is the generalized Kronecker symbol. Due to our choice of distributions for $\boldsymbol{\theta}$, the $\psi_{\boldsymbol{\alpha}}$'s are essentially Legendre polynomials and the coefficients are computed through least-squares minimization. All surrogate models are validated using 10-fold cross validation.

Polynomial chaos is by now a well documented method and we refer the readers for instance to the  UQLab \cite{uqlab} manual for more details. Incidentally, we make use of the implementation from  \cite{uqlab} in the results below. 


\subsection{Global sensitivity analysis} 
We use variance based GSA to assess the relative importance of the input parameters of $H$ in (\ref{pce}). In their simplest form,  the total Sobol' indices \cite{saltellitotalindex} apportion to uncertain parameters, or sets thereof, their relative contribution to the variance of the output. Indeed, thanks to the law of total variance, we can decompose the variance of $q = H(\hat{\boldsymbol{\theta}})$ as follows
\begin{eqnarray}
\operatorname{var}(q) = \operatorname{var}(\mathbb E[q|\hat{\boldsymbol{\theta}}_{\sim i}]) + \mathbb E[\operatorname{var}(q|\hat{\boldsymbol{\theta}}_{\sim i})], \label{ltv}
\end{eqnarray}
where $\hat{\boldsymbol{\theta}}_{\sim i}$ denotes all the parameters in $\hat{\boldsymbol{\theta}}$ except $\hat{\boldsymbol{\theta}}_i$. If we assume now that all the input parameters of $H$ are known with certainty, i.e,, if $\hat{\boldsymbol{\theta}}_{\sim i}$ is known, then the remaining variance of $q$ is simply given by 
\begin{eqnarray*}
\operatorname{var}(q) - \operatorname{var}(\mathbb E[q|\hat{\boldsymbol{\theta}}_{\sim i}]) = \mathbb E[\operatorname{var}(q|\hat{\boldsymbol{\theta}}_{\sim i})]. 
\end{eqnarray*}
This latter expression is thus a natural way of measuring how influential $\hat{\boldsymbol{\theta}}_i$ is; the corresponding total Sobol' index $S_{T_i}$ is but a scaled version of this
\begin{eqnarray}
S_{T_i} = \frac{\mathbb E[\operatorname{var}(q|\hat{\boldsymbol{\theta}}_{\sim i})]}{\operatorname{var}(q) }. \label{sobol}
\end{eqnarray}
The relevance of this basic definition can be extended to time dependent QoIs \cite{timegsa}, stochastic QoIs \cite{stogsa} or correlated parameters \cite{corvar}.


\subsection{Overall numerical approach}
The overall numerical approach combined the above three steps together with the determination and use of a corrected joint distribution discussed in \S~\ref{sec:param_sampling}; it is summarized in  Algorithm~\ref{algo}. 

\begin{algorithm}
\caption{overall numerical approach}\label{algo}
\begin{algorithmic}[1]
\While{parameter distribution has not converged}
\For{$k=1:M$} \Comment{sampling}
\State sample parameter distribution  $\longrightarrow \boldsymbol{\theta}^k$; solve (\ref{caboodle}) (without stimulus) $\longrightarrow \mathbf y^k$
\EndFor
\State estimate parameter correlations
\State fit parametrized multivariate distribution for correlated parameters; update joint distribution \Comment{see \S~\ref{sec:param_sampling}}
\EndWhile
\For{$k=1:M$} \Comment{sampling (final distribution)}
\State sample parameter distribution  $\longrightarrow \boldsymbol{\theta}^k$; solve (\ref{caboodle}) (with stimulus) $\longrightarrow \mathbf y^k$
\EndFor
\For{each QoI $q$}
\State solve the least-squares problem (\ref{lr}) \Comment{linear approximation}
\State identify the influential parameter vector $\hat{\boldsymbol\theta} =   (\theta_{j_1}, \dots, \theta_{j_r})$  \Comment{dimension reduction}
\State solve the least-squares problem  (\ref{pce}) for the polynomial  approximation $H$ \Comment{surrogate model}
\State compute total Sobol' indices (\ref{sobol}) for the surrogate model $H$
\EndFor
\end{algorithmic}
\end{algorithm}




