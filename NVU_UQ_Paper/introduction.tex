\section{Introduction}\label{sec:intro}
Over the past 20 years, the use of computational models to describe physiological phenomena has grown spectacularly. This growth has provided significant advantages to the experimental community in that it can furnish results of \textit{in silico} experiments which are either ethically or physically impossible in the laboratory. However, these models have also increased in complexity with a concomitant increase in the associated number of parameters (for a variety of reasons, most notably the requirement that the models should possess complex cellular functions which are thought \textit{but not necessarily proven} to be important). These parameters, in defining the relevant phenomenon, more often than not come from a plethora of animal experiments whose relationship to the human physiology is weak.  In addition, experimental results in the public domain provide very little information on the errors in these measurements. 

Even simple physiological models tend to be  highly non-linear; their range of applicability and reliability can only be assessed  through careful analysis. For large complex systems, the sensitivity of quantities of interest to model parameters is a priori unclear. Herein lies one of the difficulties of modeling: what effect do uncertainties in parameters determined from experiment have on the output of a non-linear numerical model? Because of both model complexity and high dimensionality, sensitivity analysis is  computationally demanding and may require several ad hoc steps--such as screening (reducing the parameter dimension). Importantly, in analysing  sensitivities, we can learn significant facts about the physiology of the system which would have stayed hidden under the premise of simply producing results. 
From a purely physiological perspective, an understanding of the dominant cellular mechanisms resulting in cerebral tissue perfusion after neuronal stimulation would be of particular and important interest. 

In the above context, we investigate the sensitivity of  the \gls{nvc} response (see Section~\ref{sec:model}), which has a large parameter dimension where most (if not all) of the parameter values come from non-human experiment  with an inherent (unknown) error. We denote by $\mathbf y = (y_1, \dots, y_N)$  the {\sl state variables} of the model and by $\boldsymbol{\theta} = (\theta_1, \dots, \theta_P)$ the {\sl uncertain parameters} of the model. The evolution of the state variables is governed by a system of ordinary differential equations (ODEs) 
\begin{eqnarray}
\frac {d\boldsymbol{y}}{dt} = \mathbf{f}(\mathbf{y}, \boldsymbol{\theta}), \label{caboodle}
\end{eqnarray}
where $\mathbf{f}$ is a known function of its arguments. Equation (\ref{caboodle}) is completed with a set of initial conditions $\boldsymbol{y}(0) = \boldsymbol{y}_0$;  here, we simply take $\boldsymbol{y}_0$ as the  equilibrium solution at the parameters' nominal values, i.e., $\mathbf f(  \mathbf{y}_0, \bar{\boldsymbol{\theta}}) = 0$, where $ \bar{\boldsymbol{\theta}} = (\bar\theta_1, \dots, \bar\theta_P)$ denotes the nominal values of the parameters. The Supplementary Material contains code and the nominal values of parameters, further information can be found in \cite{Dormanns2015a}.

Based on physiological considerations, we examine three quantities of interest (QoI), see Section~\ref{sec:model}.  Let $q$ be one of our three  QoIs; while determined from the state variables, i.e., from $\mathbf y$, $q$ is  ultimately a function of the parameters alone (and possibly time), i.e.
\begin{eqnarray}
q = g(\boldsymbol{\theta}). \label{qoi}
\end{eqnarray}
The overall goal of our numerical study is to determine  which of the uncertain parameters $\theta_1, \dots, \theta_P$  are the most/least influential for each QoI. There is a substantial amount of research currently being done in applied mathematics and statistics in the corresponding field of Global Sensitivity Analysis  (GSA). How to meaningfully define ``influential" or ``non-influential" and  how to develop methods applicable to high-dimensional problems are two significant challenges of the field \cite{corvar,timegsa,stogsa,iooss,owen,saltelli}. 



The present \gls{nvc} model contains $N=67$ state variables and $P= 160$ uncertain parameters. The complexity of the model and large parameter space dimension preclude a direct application of GSA tools. Part of our contribution in this paper is to show how multiple GSA tools may be combined to analyse such problems.

